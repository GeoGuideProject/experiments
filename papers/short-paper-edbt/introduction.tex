\section{Introduction}\label{sec:intro}

Nowadays, there exists huge amounts of spatiotemporal data in various fields such as agriculture, transportation and social science. This enables a unique opportunity to analyze and discover insights which contribute to decision makings. 

Challenges of analyzing spatiotemporal data includes discover visual patterns and trends. The visual recognition of insights is a human task which is infeasible to be automatized due to its subjectivity. Hence a visualization guidance can facilitate the process of insight discovery for a analyst. \cite{RoddickEHPS04} and \cite{Telang:2012} discuss challenges, open issues and directions considering spatiotemporal databases. From the set of challenges, it includes (i) creating and managing complex spatiotemporal simulation models, (ii) new generic temporal data models and (iii) spatiotemporal data mining, between others. 

...
...
...

Despite the progress for new spatiotemporal approaches in recent years, current analysis and visualization systems have following drawbacks: \textit{(i) Genericness} - Often systems are good for one type of spatiotemporal data and not for others. For instance, Tableau is a powerful tool [reference] for one-shot visualizations but is inefficient for filtering and multi-shot visualizations. There is a need for a generic tool which can easily capture different types of spatiotemporal data from agriculture to transportation to flights to social science; and \textit{(ii) Guidance} - As the spatiotemporal data becomes bigger and bigger, the analyst may become overwhelmed with the gigantic amount of information. Hence there is a need to guide the analyst through options which may be of her interest. The “guidance” direction has been poorly addressed in the literature.


In this paper, we propose a generic interactive analysis system which is able to guide analyst towards their interests. The guidance occurres by highlighting points which are similar to analyst choices and are as diverse as possible, so that the analysis can consider different analysis directions.

 
