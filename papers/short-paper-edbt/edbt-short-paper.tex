	% This is "sig-alternate.tex" V2.1 April 2013
% This file should be compiled with V2.5 of "sig-alternate.cls" May 2012
%
% This example file demonstrates the use of the 'sig-alternate.cls'
% V2.5 LaTeX2e document class file. It is for those submitting
% articles to ACM Conference Proceedings WHO DO NOT WISH TO
% STRICTLY ADHERE TO THE SIGS (PUBS-BOARD-ENDORSED) STYLE.
% The 'sig-alternate.cls' file will produce a similar-looking,
% albeit, 'tighter' paper resulting in, invariably, fewer pages.
%
% ----------------------------------------------------------------------------------------------------------------
% This .tex file (and associated .cls V2.5) produces:
%       1) The Permission Statement
%       2) The Conference (location) Info information
%       3) The Copyright Line with ACM data
%       4) NO page numbers
%
% as against the acm_proc_article-sp.cls file which
% DOES NOT produce 1) thru' 3) above.
%
% Using 'sig-alternate.cls' you have control, however, from within
% the source .tex file, over both the CopyrightYear
% (defaulted to 200X) and the ACM Copyright Data
% (defaulted to X-XXXXX-XX-X/XX/XX).
% e.g.
% \CopyrightYear{2007} will cause 2007 to appear in the copyright line.
% \crdata{0-12345-67-8/90/12} will cause 0-12345-67-8/90/12 to appear in the copyright line.
%
% ---------------------------------------------------------------------------------------------------------------
% This .tex source is an example which *does* use
% the .bib file (from which the .bbl file % is produced).
% REMEMBER HOWEVER: After having produced the .bbl file,
% and prior to final submission, you *NEED* to 'insert'
% your .bbl file into your source .tex file so as to provide
% ONE 'self-contained' source file.
%
% ================= IF YOU HAVE QUESTIONS =======================
% Questions regarding the SIGS styles, SIGS policies and
% procedures, Conferences etc. should be sent to
% Adrienne Griscti (griscti@acm.org)
%
% Technical questions _only_ to
% Gerald Murray (murray@hq.acm.org)
% ===============================================================
%
% For tracking purposes - this is V2.0 - May 2012

\documentclass{sig-alternate-05-2015}


\begin{document}

% Copyright
\setcopyright{acmcopyright}
%\setcopyright{acmlicensed}
%\setcopyright{rightsretained}
%\setcopyright{usgov}
%\setcopyright{usgovmixed}
%\setcopyright{cagov}
%\setcopyright{cagovmixed}


% DOI
%\doi{10.475/123_4}

% ISBN
%\isbn{123-4567-24-567/08/06}

%Conference
\conferenceinfo{EDBT '17}{March 21-24, 2017 - Venice, Italy}

%\acmPrice{\$15.00}

%
% --- Author Metadata here ---
\conferenceinfo{EDBT}{'17 Venice, Italy}
%\CopyrightYear{2007} % Allows default copyright year (20XX) to be over-ridden - IF NEED BE.
%\crdata{0-12345-67-8/90/01}  % Allows default copyright data (0-89791-88-6/97/05) to be over-ridden - IF NEED BE.
% --- End of Author Metadata ---

\title{GeoHighlight: A Point-Recommendation Approach for Spatiotemporal Data}
%\subtitle{[Extended Abstract]
%\titlenote{A full version of this paper is available as
%\textit{Author's Guide to Preparing ACM SIG Proceedings Using
%\LaTeX$2_\epsilon$\ and BibTeX} at
%\texttt{www.acm.org/eaddress.htm}}}
%
% You need the command \numberofauthors to handle the 'placement
% and alignment' of the authors beneath the title.
%
% For aesthetic reasons, we recommend 'three authors at a time'
% i.e. three 'name/affiliation blocks' be placed beneath the title.
%
% NOTE: You are NOT restricted in how many 'rows' of
% "name/affiliations" may appear. We just ask that you restrict
% the number of 'columns' to three.
%
% Because of the available 'opening page real-estate'
% we ask you to refrain from putting more than six authors
% (two rows with three columns) beneath the article title.
% More than six makes the first-page appear very cluttered indeed.
%
% Use the \alignauthor commands to handle the names
% and affiliations for an 'aesthetic maximum' of six authors.
% Add names, affiliations, addresses for
% the seventh etc. author(s) as the argument for the
% \additionalauthors command.
% These 'additional authors' will be output/set for you
% without further effort on your part as the last section in
% the body of your article BEFORE References or any Appendices.

\numberofauthors{3} %  in this sample file, there are a *total*
% of EIGHT authors. SIX appear on the 'first-page' (for formatting
% reasons) and the remaining two appear in the \additionalauthors section.
%

\author{
% You can go ahead and credit any number of authors here,
% e.g. one 'row of three' or two rows (consisting of one row of three
% and a second row of one, two or three).
%
% The command \alignauthor (no curly braces needed) should
% precede each author name, affiliation/snail-mail address and
% e-mail address. Additionally, tag each line of
% affiliation/address with \affaddr, and tag the
% e-mail address with \email.
%
% 1st. author
\alignauthor
Behrooz Omidvar-Tehrani\\
       \affaddr{Ohio State University}\\
       \affaddr{Columbus, USA}\\
       \email{omidvartehrani.1@osu.edu}
% 2nd. author
\alignauthor
Gustavo Guerino\\
       \affaddr{Federal Institute of Rio Grande do Norte - IFRN}\\
       \affaddr{Natal, Brazil}\\
       \email{gustavo.guerino@\\academico.ifrn.edu.br}
% 3rd. author
\alignauthor   Pl\'acido A. Souza Neto\\
      \affaddr{Federal Institute of Rio Grande do Norte - IFRN}\\
       \affaddr{Natal, Brazil}\\
       \email{placido.neto@ifrn.edu.br}
}
% There's nothing stopping you putting the seventh, eighth, etc.
% author on the opening page (as the 'third row') but we ask,
% for aesthetic reasons that you place these 'additional authors'
% in the \additional authors block, viz.
\additionalauthors{Additional authors: John Smith (The Th{\o}rv{\"a}ld Group,
email: {\texttt{jsmith@affiliation.org}}) and Julius P.~Kumquat
(The Kumquat Consortium, email: {\texttt{jpkumquat@consortium.net}}).}
\date{30 July 1999}
% Just remember to make sure that the TOTAL number of authors
% is the number that will appear on the first page PLUS the
% number that will appear in the \additionalauthors section.

\maketitle
\begin{abstract}
...
We are proposing a framework towards to visualize any spatio-temporal data. Although the focus of our proposal is on urban data, the framework will was designed in a way to support other types of data such as flight data, bike data, smartphone GPS data, etc.

\end{abstract}



% We no longer use \terms command
%\terms{Theory}

\keywords{Interactive analysis; Spatio-temporal visualization; Urban data.}

\section{Introduction}\label{sec:intro}

Nowadays, there exists huge amounts of spatiotemporal data in various fields such as agriculture, transportation and social science. This enables a unique opportunity to analyze and discover insights which contribute to decision makings. 

Challenges of analyzing spatiotemporal data includes discover visual patterns and trends. The visual recognition of insights is a human task which is infeasible to be automatized due to its subjectivity. Hence a visualization guidance can facilitate the process of insight discovery for a analyst. \cite{RoddickEHPS04} and \cite{Telang:2012} discuss challenges, open issues and directions considering spatiotemporal databases. From the set of challenges, it includes (i) creating and managing complex spatiotemporal simulation models, (ii) new generic temporal data models and (iii) spatiotemporal data mining, between others. 

...
...
...

Despite the progress for new spatiotemporal approaches in recent years, current analysis and visualization systems have following drawbacks: \textit{(i) Genericness} - Often systems are good for one type of spatiotemporal data and not for others. For instance, Tableau is a powerful tool [reference] for one-shot visualizations but is inefficient for filtering and multi-shot visualizations. There is a need for a generic tool which can easily capture different types of spatiotemporal data from agriculture to transportation to flights to social science; and \textit{(ii) Guidance} - As the spatiotemporal data becomes bigger and bigger, the analyst may become overwhelmed with the gigantic amount of information. Hence there is a need to guide the analyst through options which may be of her interest. The “guidance” direction has been poorly addressed in the literature.


In this paper, we propose a generic interactive analysis system which is able to guide analyst towards their interests. The guidance occurres by highlighting points which are similar to analyst choices and are as diverse as possible, so that the analysis can consider different analysis directions.

 


\section{Data Model}\label{sec:data-model}
A spatiotemporal dataset contains a vast variety of datasets such as aviation, ground transportation (bike, taxi, renting-car, bus), urban data, geo-tagged social networks, crimes, events, etc. Intuitively, the common point between all those dataset is having {\em location} and {\em time} attributes. We propose a generic data model to capture all diverse aspects of such data. 

We consider a spatiotemporal database ${\cal D}$ consisting $\langle {\cal P}, {\cal A} \rangle$ where ${\cal P}$ is the set of
geographical points and ${\cal A}$ is the set of point attributes. For each $p \in {\cal P}$, we consider a tuple $<id, lat, lon, alt, t>$ where $id$ is the point identifier, $lat$, $lon$ and $alt$ denote $p$'s geographical coordinates (latitude, longitude and altitude respectively), and $t$ is the timestamp.

The set ${\cal A}_p$ contains attribute-values for $p$ over the schema of ${\cal A}$. For instance, on a bike-sharing dataset, ${\cal A}_p = \langle $ {\tt female}, {\tt young}, {\tt subscribed} $\rangle$ on the schema ${\cal A} = \langle$ {\tt gender}, {\tt age-category}, {\tt subscription} $\rangle$ denotes that $p$ is associated to a young female cyclist who is subscribed in the bike-sharing system. The set ${\cal A}$ is domain-dependent and defines the semantics of a spatiotemporal dataset. For instance, in case of a taxi dataset, ${\cal A} = \langle$ {\tt dropoff\_time}, {\tt price}, {\tt tip} $\rangle$, where for an aviation dataset, ${\cal A} = \langle$ {\tt aircraft\_type}, {\tt departure\_airport}, {\tt arrival\_airport} $\rangle$.

Some spatiotemporal datasets contain point-sets as entities, such as {\em trajectories} in transportation datasets and {\em regions} in urban or agriculture dataset. Although our generic data model only captures the finest granular concept (i.e., point), we define ${\cal S}$ containing point-sets. Each point-set $s \in {\cal S}$ is indeed a set of points where $s \subseteq {\cal P}$. For instance, in a taxi dataset, $s = [ p_1, p_2 \dots p_n ]$ shows a ride consisting $n$ points departing at $p_1$ and arriving at $p_n$.

\section{Algorithm}\label{sec:algorithm}

Assume n points are currently shown on the map. The analysis clicks on one point p1. p1 has a set of attributes associated to it in A. Based on this set of attributes, a similarity can be measured between p1 and ant other n points. It can be for instance Cosine or Euclidean. Then the ordered list specifies the most similar points to p1. Basically we can highlight first k points in this ordered list. However, these top-k points may be too much close to each other so that it doesn’t give that much insights to the analyst.

We sacrifice similarity to gain diversity. In a greedy approach, we investigate on the k+1, k+2 … points in the ordered list and verify if the diversity increases. We can define diversity as the average/sum of distances between points. We consider a time limit (user given) and during this time limit, we try to move towards more diversity. Once the time limit expires, we show final k points.

This idea is based on \cite{Omidvar-Tehrani:2015}.

\section{Scenarios}\label{sec:scenarios}

An analyst has different goals in mind, once he needs analyzing data. Considering the spatiotemporal dataset types aforementioned, we illustrate some analyst needs and challenges with the following two realistic scenarios.

\textit{Scenario 1 - (Taxi dataset)}: Lucas is a taxi driver which works in New York city and wants to see if it is possible to improve his financial revenues during the week. He needs to choose the potential better points <p1, p2, ..., pn> in a neighborhood n1 to stay considering each weekday and hour of work. He also wants to verify, once he dropoff an client, which are the closest points that can offer him a new potential trip traject back to n1. For instance, Lucas finishes a trip traject from a point p1 in a neighborhood n1 to a point p2 in a neighborhood n2.  So, given the point p2 in n2, which are the potential closest points <p1, p2, ..., pn> (considering the day/hours of the dropoff) that can bring him back to n1 with a new client in a new trip.    

\textit{Scenario 2 (Flight dataset)}: Shadi is a flight analyst and she has to propose new solutions for improve airport and flights performances. She wants to know: (i) the best time of day/day of week/time of year to fly to minimise delays; (ii) How does the number of people flying between different locations change over time? And (iii) Is it possible to detect cascading failures as delays in one airport create delays in others? And considering this, propose a solution previously. 


\section{Related Works}\label{sec:relatedworks}
\section{Conclusions}\label{sec:conclusions}



%
% The following two commands are all you need in the
% initial runs of your .tex file to
% produce the bibliography for the citations in your paper.
\bibliographystyle{abbrv}
\bibliography{biblio} 

\end{document}
