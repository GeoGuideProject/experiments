% This is "sig-alternate.tex" V2.1 April 2013
% This file should be compiled with V2.5 of "sig-alternate.cls" May 2012
%
% This example file demonstrates the use of the 'sig-alternate.cls'
% V2.5 LaTeX2e document class file. It is for those submitting
% articles to ACM Conference Proceedings WHO DO NOT WISH TO
% STRICTLY ADHERE TO THE SIGS (PUBS-BOARD-ENDORSED) STYLE.
% The 'sig-alternate.cls' file will produce a similar-looking,
% albeit, 'tighter' paper resulting in, invariably, fewer pages.
% 
% ----------------------------------------------------------------------------------------------------------------
% This .tex file (and associated .cls V2.5) produces:
%       1) The Permission Statement
%       2) The Conference (location) Info information
%       3) The Copyright Line with ACM data
%       4) NO page numbers
%
% as against the acm_proc_article-sp.cls file which
% DOES NOT produce 1) thru' 3) above.
%
% Using 'sig-alternate.cls' you have control, however, from within
% the source .tex file, over both the CopyrightYear
% (defaulted to 200X) and the ACM Copyright Data
% (defaulted to X-XXXXX-XX-X/XX/XX).
% e.g.
% \CopyrightYear{2007} will cause 2007 to appear in the copyright line.
% \crdata{0-12345-67-8/90/12} will cause 0-12345-67-8/90/12 to appear in the copyright line.
%
% ---------------------------------------------------------------------------------------------------------------
% This .tex source is an example which *does* use
% the .bib file (from which the .bbl file % is produced).
% REMEMBER HOWEVER: After having produced the .bbl file,
% and prior to final submission, you *NEED* to 'insert'
% your .bbl file into your source .tex file so as to provide
% ONE 'self-contained' source file.
%
% ================= IF YOU HAVE QUESTIONS =======================
% Questions regarding the SIGS styles, SIGS policies and
% procedures, Conferences etc. should be sent to
% Adrienne Griscti (griscti@acm.org)
%
% Technical questions _only_ to
% Gerald Murray (murray@hq.acm.org)
% ===============================================================
%
% For tracking purposes - this is V2.0 - May 2012

\documentclass{sig-alternate-05-2015}
\usepackage{algorithm} 
\usepackage{algorithmic}

\begin{document}

% Copyright
\setcopyright{acmcopyright}

%\setcopyright{acmlicensed}
%\setcopyright{rightsretained}
%\setcopyright{usgov}
%\setcopyright{usgovmixed}
%\setcopyright{cagov}
%\setcopyright{cagovmixed}

 
% DOI
\doi{10.475/123_4}

% ISBN
\isbn{123-4567-24-567/08/06}

%Conference
\conferenceinfo{PLDI '13}{June 16--19, 2013, Seattle, WA, USA}

\acmPrice{\$15.00}

%
% --- Author Metadata here ---
\conferenceinfo{WOODSTOCK}{'97 El Paso, Texas USA}
%\CopyrightYear{2007} % Allows default copyright year (20XX) to be over-ridden - IF NEED BE.
%\crdata{0-12345-67-8/90/01}  % Allows default copyright data (0-89791-88-6/97/05) to be over-ridden - IF NEED BE.
% --- End of Author Metadata ---

\title{GeoHighlight: A Point-Recommendation Approach for Spatiotemporal Data}
%\subtitle{[Extended Abstract]
%\titlenote{A full version of this paper is available as
%\textit{Author's Guide to Preparing ACM SIG Proceedings Using
%\LaTeX$2_\epsilon$\ and BibTeX} at
%\texttt{www.acm.org/eaddress.htm}}}
%
% You need the command \numberofauthors to handle the 'placement
% and alignment' of the authors beneath the title.
%
% For aesthetic reasons, we recommend 'three authors at a time'
% i.e. three 'name/affiliation blocks' be placed beneath the title.
%
% NOTE: You are NOT restricted in how many 'rows' of
% "name/affiliations" may appear. We just ask that you restrict
% the number of 'columns' to three.
%
% Because of the available 'opening page real-estate'
% we ask you to refrain from putting more than six authors
% (two rows with three columns) beneath the article title.
% More than six makes the first-page appear very cluttered indeed.
%
% Use the \alignauthor commands to handle the names
% and affiliations for an 'aesthetic maximum' of six authors.
% Add names, affiliations, addresses for
% the seventh etc. author(s) as the argument for the
% \additionalauthors command.
% These 'additional authors' will be output/set for you
% without further effort on your part as the last section in
% the body of your article BEFORE References or any Appendices.

\numberofauthors{3} %  in this sample file, there are a *total*
% of EIGHT authors. SIX appear on the 'first-page' (for formatting
% reasons) and the remaining two appear in the \additionalauthors section.
%

\author{
% You can go ahead and credit any number of authors here,
% e.g. one 'row of three' or two rows (consisting of one row of three
% and a second row of one, two or three).
%
% The command \alignauthor (no curly braces needed) should
% precede each author name, affiliation/snail-mail address and
% e-mail address. Additionally, tag each line of
% affiliation/address with \affaddr, and tag the
% e-mail address with \email.
%
% 1st. author
\alignauthor
Behrooz Omidvar-Tehrani\\
       \affaddr{Ohio State University}\\
       \affaddr{Columbus, USA}\\
       \email{omidvartehrani.1@osu.edu}
% 2nd. author
\alignauthor
Gustavo Guerino\\
       \affaddr{Federal Institute of Rio Grande do Norte - IFRN}\\
       \affaddr{Natal, Brazil}\\
       \email{gustavo.guerino@\\academico.ifrn.edu.br}
% 3rd. author
\alignauthor   Pl\'acido A. Souza Neto\\
      \affaddr{Federal Institute of Rio Grande do Norte - IFRN}\\
       \affaddr{Natal, Brazil}\\
       \email{placido.neto@ifrn.edu.br}
}
% There's nothing stopping you putting the seventh, eighth, etc.
% author on the opening page (as the 'third row') but we ask,
% for aesthetic reasons that you place these 'additional authors'
% in the \additional authors block, viz.

\date{30 July 1999}
% Just remember to make sure that the TOTAL number of authors
% is the number that will appear on the first page PLUS the
% number that will appear in the \additionalauthors section.

\maketitle
\begin{abstract}

This paper proposes an approach towards generic visualization of
spatiotemporal data. Although the initial focus of our proposal is on urban
data, our framework was designed in a way to support any types of
\textit{geo} data such as flight data, bike data, smartphone GPS data, etc.
Highlighting spatiotemporal data is also a challenge that this work exploits, by using
recommendation algorithm to produce relevant results from datasets that has
geographic informations. We also discuss some directions and benefits of using
recommendation algorithm to highlight or retrieve important informations from
these kind of datasets.

\end{abstract}



% We no longer use \terms command
%\terms{Theory}

\keywords{Interactive analysis; Spatio-temporal visualization; Urban data.}

\section{Introduction} 

Nowadays, there exists huge amounts of spatiotemporal data in various fields
such as agriculture, transportation and social science. This enables
a unique opportunity to analyze and discover insights which contribute to
decision makings.    

Challenges of analyzing spatiotemporal data includes discover visual patterns
and trends. The visual recognition of insights is a human task which is
infeasible to be automatized due to its subjectivity. Hence a visualization
guidance can facilitate the process of insight discovery for a analyst.
\cite{RoddickEHPS04} and \cite{Telang:2012} discuss challenges, open issues and
directions considering spatiotemporal databases. From the set of challenges, it
includes (i) creating and managing complex spatiotemporal simulation models,
(ii) new generic temporal data models and (iii) spatiotemporal data mining,
between others.         


Despite the progress for new spatiotemporal approaches in recent years, current
analysis and visualization systems have following drawbacks: \textit{(i)
Genericness} - Often systems are good for one type of spatiotemporal data and
not for others. For instance, Tableau is a powerful tool [reference] for
one-shot visualizations but is inefficient for filtering and multi-shot
visualizations. There is a need for a generic tool which can easily capture
different types of spatiotemporal data from agriculture to transportation to
flights to social science; and \textit{(ii) Guidance} - As the spatiotemporal
data becomes bigger and bigger, the analyst may become overwhelmed with the
gigantic amount of information. Hence there is a need to guide the analyst
through options which may be of her interest. The ``\textit{guidance}''
direction has been poorly addressed in the literature.            


In this paper, we propose a generic interactive analysis system which is able to
guide analyst towards their interests on spatiotemporal informations. The
guidance occurres by highlighting points which are similar to analyst choices and are as diverse as possible, so
that the analysis can consider different analysis directions. For that, this
paper merges a recommendation algorithm approach \cite{Omidvar-Tehrani:2015} and
visual highlighting proposals \cite{Lohmann:2012,Robinson2011,Liang2010} as the
bases towards a generic framework to guide spatiotemporal analysis.

The outline of the paper is as follows: in Section \ref{sec:data-model} we
present the formalization of the proposed Data Model for spatiotemporal
highlight. In Section \ref{sec:problem-definition} we describe the problem we
are attacking. Section \ref{sec:algorithm} presents and formilizes the proposed
algorithm. The scenarios and related work is provided in Sections \ref{sec:scenarios} and
\ref{sec:related-works}. Finally, in Section \ref{sec:conclusions}, we present 
some conclusions and perspectives for future works.

\section{Data Model}\label{sec:data-model}
We consider a database D of spatiotemporal data $<P,A>$ where P is the set of
points and A is the set of point attributes. P contains tuples of the form
$<pid, lat, lng>$ where pid is the point identifier and lat and lng are latitude
and longitude of the point. Also A contains tuples in the form $<pid, a1,
a2,..., an>$ where a1 ... an are the set of attributes. For instance, in case of
taxi data set, A is like$< pid, dropoff\_time, price, tip>$. In case of flight data, A is like $pid, altitude, speed$.

The database can also have a set of relations $<R>$ between two points $p1,p2
\in P$. For instance,  a relation $r1 \in R$ can represent a taxi trip, where p1
is the pickup point, and p2 is the dropoff point.    
...
\section{Problem Definition}\label{sec:problem-definition}

Considering an exploratory analysis context, we address the following question:
``\textit{how does a spatiotemporal visualization tool guide the analyst towards
interesting regions in an efficient and personalized matter?}'' This is a
challenges question due to customizability and scalability issues. First an
analyst-tailored recommendation is subjective and hard to capture with an
unsupervised method. Second, such recommendation needs a thorough consideration
of all points which is time-consuming.       

We describe our problem setting in form a scenario. Consider n points are
currently shown on the map where n is the size of the dataset. The analyst is
interested in the point p1 and clicks on it. p1 has a set of attributes
associated to it in A. Hance a similarity value can be measured between p1 and
any other n-1 points. The k-most similar points can then be highlighted for
guiding the analyst what to see next. However, those k points may be too much
close to each other so that it doesn't give that much insights to the analyst.
Hence, it is ideal to slightly sacrifice similarity to gain diversity. Finally,
the analyst can see k points which are maximally similar to the point of her
interest and they are as diverse as possible throughout the region of
investigation. We formally define our problem as follows.            

Problem: Given an input point p, the GeoHighlight problem is to return k points referref as Pk which satisfy the following conditions:

1. Pk respects similarity
2. |Pk|=k
3. diversity(Pk) is maximized.

where diversity(Pk) is defined as follows: diversity(Pk) = [BEHROOZ WILL COMPLETE]

We prove that our problem is NP-Complete by a reduction from MAXIMUM EDGE SUBGRAPH Problem.

\section{Algorithm}\label{sec:algorithm}

In this section, we present our GeoHighlight algorithm which can recommend k
points to be considered in the next iteration of analysis. The algorithm needs
to consider two dimensions at the same time: similarity and diversity.
Similarity guarantees that the current analysis context will be preserved and
the next step will be the natural advancement of the current step. Diversity
ensures that different aspects of data will be covered in the next step so that
the analyst obtains a global view.       

As the analyst investigates on maps in an interactive manner, long delays are
not accepted because they break the analyst's train of thoughts. Hence the
algorithm should operate fast. Fro this aim, the challenge is to efficiently
compute similarity and diversity values. Although different formulations of
these two metrics (Jaccard, Cosine, etc.) may enforce different time
complexities, another challenge is to pick a formulation that semantically makes
sense and can guide the analyst to fruitful directions.      

GeoHighlight is illustrated in Algorithm \ref{algo:geo-highlight}
which is an adaption of the algorithm in \cite{Omidvar-Tehrani:2015}. The
algorithm operates in two consecutive stages: off-line process to prepare
indexes and online process to select k most relevant and diverse points.    

\begin{algorithm}
\small
\label{algo:geo-highlight}
\begin{algorithmic}
\REQUIRE
\ENSURE
\end{algorithmic}

\caption{ - \textit{GeoHighlight} Algorithm}

\end{algorithm}   

We consider an offline step to produce indexes for each single geographical
point in our dataset to boost the online execution. An index for a point p
contains the similarity value between p and all other points in decreasing
order. For instance, at the head of the index for the point p where [CONSIDER AN
EXAMPLE POINT HERE], the following point is observed [POINT]. We use a parameter
$\mu$  to partially materialize the indexes: points whose similarity with the
 input point is lower than $\mu$ will not be stored in index.        

The online step is a greedy procedure which admits as input the current point
under investigation and returns the best k points as output. The algorithm first
looks at the top-k points in the index of the point p. Function getNext()
returns the next point in the index in sequential order. The algorithm iterates
over the index to determine if other points should be considered to increase
diversity while staying within the time limit and not violating the similarity
threshold μ. Since points in the index are sorted on decreasing similarity value
with the point p, the algorithm can safely stop as soon as the similarity
condition is violated (or if the time limit is exceeded).        

The algorithm then looks for a candidate point to replace in order to increase
diversity. The boolean function betterDiv() checks if by this replacement, the
overall diversity increases.    

The number of diversity improvement loops is equal to size of the index in worst
case. For each point, we verify if the diversity score is improved by
betterDiv(), hence O(k2). The time complexity of the algorithm is then
O(k2.|P|).     

\section{Scenarios}\label{sec:scenarios}

An analyst has different goals in mind once she needs analyzing data.
Considering the spatiotemporal dataset types aforementioned, we illustrate some
analyst's needs and challenges with the following two realistic scenarios.   
\bigskip

\textit{Scenario 1 - Taxi dataset:} Lucas is a taxi driver which works in New
York city. He wants to see if it is possible to improve his financial revenues
during the week. One option to Lucas may be to choose the potential better
points [$p_1, p_2, \ldots, p_n $] in a neighborhood [$n_1$] to stay considering
a week day and the time of work. He also wants to verify, once he dropoff an
client, which are the closest points [$cp_1, cp_2, \ldots, cp_n $] that can
offer him a new potential trip trajet back to [$n1$]. For instance, Lucas
finishes a trip trajet from a point $p_1$ in a neighborhood [$n_1$] to a point
$p_2$ in a neighborhood $n_2$.  So, given the point p2 in n2, which are the
potential closest points [$cp_1, cp_2, \ldots, cp_n $] (considering the
day/hours of the dropoff) that can bring him back to [$n_1$] with a new client
in a new trip with the small possible time of waiting.
\bigskip

\textit{Scenario 2 - Flight dataset}: Shadi is a flight analyst and she has to
propose new solutions for improve airport and flights performances. She wants to know:
(i) the best time of day/day of week/time of year to fly to minimise delays;
(ii) How does the number of people flying between different locations change
over time? And (iii) Is it possible to detect cascading failures as delays in
one airport create delays in others? And considering this fact, propose a
solution previously. In the Shadi's working days she needs to propose quickly
solutions considering the mentioned aspects. 
 
% Table \ref{table:1} shows the operations for both scenarios.
% 
% \begin{table}[h!]
% \centering
% \begin{tabular}{||c c c ||} 
%  \hline
%  Col1 & Taxi scenario & Taxi scenario  \\ [0.5ex] 
%  \hline\hline
%  1 & 6 & 87837  \\ 
%  2 & 7 & 78  \\
%  3 & 545 & 778 \\
%  4 & 545 & 18744  \\
% 
%  \hline
% \end{tabular}
% \caption{Scenarios}
% \label{table:1}
% \end{table}



\section{Related Work}\label{sec:related-works}

In this paper, we addressed the problem of highlighting geographical points to
guide analysts in consecutives steps. To the best of our knowledge, this is the
first work which formulates the geo-highlighting problem for recommending on
maps. However, our problem relates to a number of others as follows.     

Visual Highlighting. There has been efforts to highlight some pieces of
information in the huge heterogeneous data space so that the analyst can focus
on important aspects \cite{Liang2010,Lohmann:2012,Robinson2011}. Visual
Story Construction is another domain of work \cite{Segel:2010,Samet:2014} which
consecutive analysis steps are created automatically. However, all such highlighting methods are objective and does not consider analyst
interest. In GeoHiglight, we provide a simple yet effective feedback model which
can feed the recommendation algorithm to produce relevant results to current
investigations.        

Some other works have exploited highlighting as a technique to synchronize
coordinated views [CITE CROSSFILTER] or simplify complicated dataset
visualizations \cite{Robinson2011,Alper:2011}. Also in \cite{Philipsen}
different highlighting methods are compared in terms of efficiency and usefulness. Such methods are complementary to ours.     

Spatiotemporal Interactive Analysis and Visualization. Being an interactive
system, it should be efficient and capture user feedback and adapt the utility
function. In terms of efficiency, SpatialHadoop \cite{} and GeoSpark \cite{}
extend Hadoop and Spark ecosystems respectively to boost geographical
computations and visualizations. Such systems can be exploited as the backbone
of GeoHighlight once large-scale data needs to be analyzed.      

In terms of feedback, many off-the-shelf products such as Tableau \cite{} and
RapidMiner \cite{} are designed to visualize different kinds of datasets including
spatiotemporal ones. However, ``recommendation'' is the missing component in
most of such tools: providing a full package of operations and actions, the analyst
may know what to do next. In such system, analysis is usually considered as a
one-shot scenario, once in reality it happens in consecutive steps following
user feedback.        

Recommendation. There exist a huge body of work in recommendation
\cite{Adomavicius:2005} for various domains and datasets. However, spatial
dimension is still untouched and has not received a lot of attention. In
\cite{ChirigatiDDF16} a prediction algorithm for urban data is introduced which operates offline. In
\cite{Levandoski:2012,Magdy2014,HendawiKRBTA15a,Bao2015,Magdy:2014},
recommendation and visualization tools are introduced in specific domains.
However, most the such algorithms are not context-aware (based on user choices)
and does not consider diversity as a global metric.    

\section{Conclusions}\label{sec:conclusions}


\bibliographystyle{abbrv}
\bibliography{biblio} 

\end{document}
