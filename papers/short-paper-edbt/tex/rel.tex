\section{Related Work}
\label{sec:rel}

% In this paper, we addressed the problem of highlighting geographical points to
% guide analysts in consecutives steps. To the best of our knowledge, this is the
% first work which formulates the geo-highlighting problem for recommending on
% maps. However, our problem relates to a number of others as follows.     

% Visual Highlighting. There has been efforts to highlight some pieces of
% information in the huge heterogeneous data space so that the analyst can focus
% on important aspects \cite{Liang2010,Lohmann:2012,Robinson2011}. Visual
% Story Construction is another domain of work \cite{Segel:2010,Samet:2014} which
% consecutive analysis steps are created automatically. However, all such highlighting methods are objective and does not consider analyst
% interest. In GeoHiglight, we provide a simple yet effective feedback model which
% can feed the recommendation algorithm to produce relevant results to current
% investigations.        

% Some other works have exploited highlighting as a technique to synchronize
% coordinated views [CITE CROSSFILTER] or simplify complicated dataset
% visualizations \cite{Robinson2011,Alper:2011}. Also in \cite{Philipsen}
% different highlighting methods are compared in terms of efficiency and usefulness. Such methods are complementary to ours.     

% Spatiotemporal Interactive Analysis and Visualization. Being an interactive
% system, it should be efficient and capture user feedback and adapt the utility
% function. In terms of efficiency, SpatialHadoop \cite{} and GeoSpark \cite{}
% extend Hadoop and Spark ecosystems respectively to boost geographical
% computations and visualizations. Such systems can be exploited as the backbone
% of GeoHighlight once large-scale data needs to be analyzed.      

% In terms of feedback, many off-the-shelf products such as Tableau \cite{} and
% RapidMiner \cite{} are designed to visualize different kinds of datasets including
% spatiotemporal ones. However, ``recommendation'' is the missing component in
% most of such tools: providing a full package of operations and actions, the analyst
% may know what to do next. In such system, analysis is usually considered as a
% one-shot scenario, once in reality it happens in consecutive steps following
% user feedback.        

% Recommendation. There exist a huge body of work in recommendation
% \cite{Adomavicius:2005} for various domains and datasets. However, spatial
% dimension is still untouched and has not received a lot of attention. In
% \cite{ChirigatiDDF16} a prediction algorithm for urban data is introduced which operates offline. In
% \cite{Levandoski:2012,Magdy2014,HendawiKRBTA15a,Bao2015,Magdy:2014},
% recommendation and visualization tools are introduced in specific domains.
% However, most the such algorithms are not context-aware (based on user choices)
% and does not consider diversity as a global metric.    
