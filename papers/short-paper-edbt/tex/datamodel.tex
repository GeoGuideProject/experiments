\section{Data Model}\label{sec:data-model}
% prune this
A wide range of spatiotemporal data is present in a vast variety of datasets such as aviation, ground transportation (bike, taxi, renting- car, bus), urban data, geo-tagged social networks, crimes, events, etc. Intuitively, the common point between all those dataset is having {\em location} and {\em time} attributes. Based on this particularity, we propose a generic data model to capture all diverse aspects of such data.

We consider a spatiotemporal database ${\cal D}$ consisting $\langle {\cal P}, {\cal A} \rangle$ where ${\cal P}$ is the set of
geographical points and ${\cal A}$ is the set of point attributes. For each $p \in {\cal P}$, we consider a tuple $<id, lat, lon, alt, t>$ where $id$ is the point identifier, $lat$, $lon$ and $alt$ denote $p$'s geographical coordinates (latitude, longitude and altitude respectively), and $t$ is the timestamp.

The set ${\cal A}_p$ contains attribute-values for $p$ over the schema of ${\cal A}$. For instance, on a bike-sharing dataset, ${\cal A}_p = \langle $ {\tt female}, {\tt young}, {\tt subscribed} $\rangle$ on the schema ${\cal A} = \langle$ {\tt gender}, {\tt age-category}, {\tt subscription} $\rangle$ denotes that $p$ is associated to a young female cyclist who is subscribed in the bike-sharing system. The set ${\cal A}$ is domain-dependent and defines the semantics of a spatiotemporal dataset. For instance, in case of a taxi dataset, ${\cal A} = \langle$ {\tt dropoff\_time}, {\tt price}, {\tt tip} $\rangle$, where for an aviation dataset, ${\cal A} = \langle$ {\tt aircraft\_type}, {\tt departure\_airport}, {\tt arrival\_airport} $\rangle$.

% prune this
Some spatiotemporal datasets contain point-sets as entities, such as {\em trajectories} in transportation datasets and {\em regions} in urban or agriculture dataset. Although our generic data model only captures the finest granular concept (i.e., point), we define ${\cal S}$ containing point-sets. Each point-set $s \in {\cal S}$ is indeed a set of points where $s \subseteq {\cal P}$. For instance, in a taxi dataset, $s = [ p_1, p_2 \dots p_n ]$ shows a ride consisting $n$ points departing at $p_1$ and arriving at $p_n$.