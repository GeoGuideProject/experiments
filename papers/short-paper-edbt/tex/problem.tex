\section{Problem Statement}
\label{sec:pb}
% In an exploratory analysis context, the analyst does not necessarily know what to ask. She may have also a few knowledge about the spatiotemporal data and its attributes. Hence she usually needs to take iterative analysis steps to observe different aspects of data and ultimately land on a subset of interest. However, it is often cumbersome to choose what to analyze next. Because this choice is subjective and infeasible to capture with an unsupervised method.

\noindent {\bf Data Model.} We consider a spatiotemporal database ${\cal D}$ consisting $\langle {\cal P}, {\cal A} \rangle$ where ${\cal P}$ is the set of
geographical points and ${\cal A}$ is the set of point attributes. For each $p \in {\cal P}$, we consider a tuple $\langle lat, lon, alt, t\rangle$ where $lat$, $lon$ and $alt$ denote $p$'s geographical coordinates (latitude, longitude and altitude respectively), and $t$ is the timestamp. The set ${\cal A}_p$ contains attribute-values for $p$ over the schema of ${\cal A}$. For instance, on a bike-sharing dataset, ${\cal A}_p = \langle${\tt female}, {\tt young}, {\tt hybrid-bike}$\rangle$ on the schema ${\cal A} = \langle${\tt gender}, {\tt age}, {\tt type}$\rangle$ denotes that $p$ is associated to a young female cyclist who rides a hybrid bike. The set ${\cal A}$ is domain-dependent and defines the semantics of a spatiotemporal dataset.

\vspace{5pt}
In this paper, we address the problem of {\em generic guidance} in spatiotemporal data: ``what is the process of guiding analysts in iterative analysis steps on any spatiotemporal dataset?'' In other words, we are interested in an approach which highlights a set of $k$ points that the analyst should consider in the next analysis iteration. This should not be a heuristic-based data-dependent highlighting, but a generic approach which is applied on any spatiotemporal dataset. We describe the desiderata of generic guidance approach as follows.

\vspace{5pt}
\noindent {\bf D1. Genericness.} The guidance component should be agnostic (making no assumption) about the dataset type, attributes and distribution.

\vspace{5pt}
\noindent {\bf D2. Limited Options.} The set of $k$ highlighted points should not be very large because too many options distract the analyst. % \cite{miller1956human}.

\vspace{5pt}
\noindent {\bf D3. Relevance.} The fundamental difference between highlighting and $k$-NN spatial queries \cite{aly2015spatial} is that, in the former, the focus is on $k$ points which have similar characteristics to~$p$, hence relevant.
% In other words, we are interested in points which are {\em relevant} to a given point of interest.
For instance, consider a taxi ride in New York for a young male customer for an itinerary of 10 kilometers and \$3 tip. In contrary to thousands of kilometers of geographical distance, the ride is very similar to another one in San Fransisco for a middle-age male customer for an itinerary of 8 kilometers and \$2.5 tip.
% Relevance is a pairwise metric which is associated to point characteristics.
Given two points $p$ and $p'$, we define {\em relevance} as follows.

% \begin{definition}[Relevance]
% Given two points $p$ and $p'$ and their attribute values ${\cal A}_{p}$ and ${\cal A}_{p'}$, the relevance between $p$ and $p'$ is a value between $0$ and $1$ denoted as $\mathit{relevance}(p,p') = \mathit{average}_{a \in {\cal A}_{p} \cup {\cal A}_{p'}}(\mathit{sim({\cal A}_{p}, {\cal A}_{p'}, a)})$.
% \label{def:rel}
% \end{definition}

\begin{dmath}
\label{eq:rel}
\mathit{relevance}(p,p') = \mathit{average}_{a \in {\cal A}_{p} \cup {\cal A}_{p'}}(\mathit{sim(p, p', a)})
\end{dmath}

The similarity function $\mathit{sim}()$ can be any function such as Jaccard and Cosine. Each attribute can have its own similarity function (as string and integer attributes are compared differently.) Then $\mathit{sim}()$ works as an overriding-function which provides encapsulated similarity computations for any type of attribute.

\vspace{5pt}
\noindent {\bf D4. Diversity.} A guidance approach should also consider coverage of all points: $k$ highlighted points should represent distinct regions so that the analyst can observe different aspects of data and decide for the next analysis iteration. Hence, $k$ points should be diverse.
% Diversity is a set-based metric and is associated to geographical distance. We define this metric as follows.
Given a set of points $s = \{ p_1, p_2 \dots \}$, we define {\em diversity} as follows.

\begin{dmath}
\label{eq:divs}
\mathit{diversity}(s) = \mathit{average}_{\{p, p'\} \subseteq s | p \neq p' } \mathit{distance}(p,p')
\end{dmath} 

The function $\mathit{distance}(p,p')$ operates on geographical coordinates of $p$ and $p'$ and can be considered as any distance function of Chebyshev distance family such as Eucledian. However, as distance computations are done in {\em spherical space} using latitude, longitude and altitude, it is au-naturel to employ Harvestine distance shown in Equation \ref{eq:harvestine}.

\begin{dmath}
\label{eq:harvestine}
distance(p,p') = [ acos(cos(p_{lat}) . cos(p'_{lat}) . cos(p_{lon}) . cos(p'_{lon})\\ + cos(p_{lat}) . sin(p'_{lat}). cos(p_{lon}) . sin(p'_{lon}) + sin(p_{lat}) . sin(p'_{lat})) ] \times earth\_radius
\end{dmath}

\noindent {\bf D5. Interactivity.} The exploratory nature of the analysis requires the guidance component to be involved in an interactive process. Hence the analyst can investigate and refine different aspects of spatiotemporal data in iterative steps. For being interactive, the guidance component should be efficient so that the train of thought of analyst would not be broken during the analysis process.

\vspace{5pt}
Following aforementioned desiderata, we formulate highlighting as an optimization-based problem on relevance and diversity dimensions.

\begin{problem}[\pb]
\label{pb:geoh}
Given an input point $p$ and a threshold $\sigma$, the problem is to return top-$k$
 points denoted $S_p$ where $|S_p| = k$ and $\forall p' \in S_p, \mathit{relevance}(p,p') \geq \sigma$ and $\mathit{diversity}(S_p)$ is maximized.
\end{problem}

Problem \ref{pb:geoh} is hard due to the huge space of spatiotemporal data: for any given point $p$, an exhaustive search over all other points is necessary to find $k$ points with maximal relevance. Moreover, the problem investigates in two dimensions at the same time (relevance and diversity) which makes it more challenging.

% behrooz: talk about quality earlier